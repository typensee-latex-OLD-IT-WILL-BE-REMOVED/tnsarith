%\documentclass[12pt,a4paper]{scrartcl}
\documentclass[12pt,a4paper]{article}

\makeatletter % Technical doc - START

\usepackage[utf8]{inputenc}
\usepackage[T1]{fontenc}
\usepackage{ucs}

\usepackage[french]{babel,varioref}

\usepackage[top=2cm, bottom=2cm, left=1.5cm, right=1.5cm]{geometry}
\usepackage{enumitem}

\usepackage{pgffor}

\usepackage{multicol}

\usepackage{makecell}

\usepackage{color}
\usepackage{hyperref}
\hypersetup{
    colorlinks,
    citecolor=black,
    filecolor=black,
    linkcolor=black,
    urlcolor=black
}

\usepackage{amsthm}

\usepackage{tcolorbox}
\tcbuselibrary{listingsutf8}

\usepackage{ifplatform}

\usepackage{ifthen}

\usepackage{macroenvsign}


% Sections numbering

\renewcommand\thesection{\arabic{section}.}
\renewcommand\thesubsection{\alph{subsection}.}
\renewcommand\thesubsubsection{\roman{subsubsection}.}


% MISC

\newtcblisting{latexex}{%
	sharp corners,%
	left=1mm, right=1mm,%
	bottom=1mm, top=1mm,%
	colupper=red!75!blue,%
	listing side text
}

\newtcbinputlisting{\inputlatexex}[2][]{%
	listing file={#2},%
	sharp corners,%
	left=1mm, right=1mm,%
	bottom=1mm, top=1mm,%
	colupper=red!75!blue,%
	listing side text
}


\newtcblisting{latexex-flat}{%
	sharp corners,%
	left=1mm, right=1mm,%
	bottom=1mm, top=1mm,%
	colupper=red!75!blue,%
}

\newtcbinputlisting{\inputlatexexflat}[2][]{%
	listing file={#2},%
	sharp corners,%
	left=1mm, right=1mm,%
	bottom=1mm, top=1mm,%
	colupper=red!75!blue,%
}


\newtcblisting{latexex-alone}{%
	sharp corners,%
	left=1mm, right=1mm,%
	bottom=1mm, top=1mm,%
	colupper=red!75!blue,%
	listing only
}

\newtcbinputlisting{\inputlatexexalone}[2][]{%
	listing file={#2},%
	sharp corners,%
	left=1mm, right=1mm,%
	bottom=1mm, top=1mm,%
	colupper=red!75!blue,%
	listing only
}


\newcommand\inputlatexexcodeafter[1]{%
	\begin{center}
		\input{#1}
	\end{center}

	\vspace{-.5em}
	
	Le rendu précédent a été obtenu via le code suivant.
	
	\inputlatexexalone{#1}
}


\newcommand\inputlatexexcodebefore[1]{%
	\inputlatexexalone{#1}
	\vspace{-.75em}
	\begin{center}
		\textit{\footnotesize Rendu du code précédent}
		
		\medskip
		
		\input{#1}
	\end{center}
}


\newcommand\env[1]{\texttt{#1}}
\newcommand\macro[1]{\env{\textbackslash{}#1}}



\setlength{\parindent}{0cm}
\setlist{noitemsep}

\theoremstyle{definition}
\newtheorem*{remark}{Remarque}

\usepackage[raggedright]{titlesec}

\titleformat{\paragraph}[hang]{\normalfont\normalsize\bfseries}{\theparagraph}{1em}{}
\titlespacing*{\paragraph}{0pt}{3.25ex plus 1ex minus .2ex}{0.5em}


\newcommand\separation{
	\medskip
	\hfill\rule{0.5\textwidth}{0.75pt}\hfill
	\medskip
}


\newcommand\extraspace{
	\vspace{0.25em}
}


\newcommand\whyprefix[2]{%
	\textbf{\prefix{#1}}-#2%
}

\newcommand\mwhyprefix[2]{%
	\texttt{#1 = #1-#2}%
}

\newcommand\prefix[1]{%
	\texttt{#1}%
}


\newcommand\inenglish{\@ifstar{\@inenglish@star}{\@inenglish@no@star}}

\newcommand\@inenglish@star[1]{%
	\emph{\og #1 \fg}%
}

\newcommand\@inenglish@no@star[1]{%
	\@inenglish@star{#1} en anglais%
}


\newcommand\ascii{\texttt{ASCII}}


% Example
\newcounter{paraexample}[subsubsection]

\newcommand\@newexample@abstract[2]{%
	\paragraph{%
		#1%
		\if\relax\detokenize{#2}\relax\else {} -- #2\fi%
	}%
}



\newcommand\newparaexample{\@ifstar{\@newparaexample@star}{\@newparaexample@no@star}}

\newcommand\@newparaexample@no@star[1]{%
	\refstepcounter{paraexample}%
	\@newexample@abstract{Exemple \theparaexample}{#1}%
}

\newcommand\@newparaexample@star[1]{%
	\@newexample@abstract{Exemple}{#1}%
}


% Change log
\newcommand\topic{\@ifstar{\@topic@star}{\@topic@no@star}}

\newcommand\@topic@no@star[1]{%
	\textbf{\textsc{#1}.}%
}

\newcommand\@topic@star[1]{%
	\textbf{\textsc{#1} :}%
}

\makeatother % Technical doc - END


\usepackage{tnsarith}


\begin{document}

\renewcommand\labelitemi{\raisebox{0.125em}{\tiny\textbullet}}
\renewcommand{\labelitemii}{---}

\title{  %
	Le package \texttt{tnsarith}:\\%
	un peu d'arithmétique élémentaire\\%
	{\footnotesize Code source disponible sur \url{https://github.com/typensee-latex/tnsarith.git}.}\\%
{\footnotesize Version \texttt{0.1.0-beta} développée et testée sur \macosxname{}.}%
}
\author{Christophe BAL}
\date{2020-07-12}

\maketitle


\vspace{2em}

\hrule

\tableofcontents

\vspace{1.5em}

\hrule

\newpage

\section{Introduction}

Le package \verb+tnsarith+ propose quelques macros pour rédiger un peu d'arithmétique  via un codage sémantique simple.
\section{Ensembles classiques}

Le package \verb+tnssets+ propose les macros \macro{NN}, \macro{ZZ}, \macro{QQ} ainsi que \macro{PP} pour indiquer l'ensemble des naturels, celui des entiers relatifs, celui des fractions rationnelles et enfin celui des nombres premiers.
Se rendre sur \url{https://github.com/typensee-latex/tnssets.git} si cela vous intéresse.
\section{Opérateurs de base}

Pour des raisons d'expressivité des codes \LaTeX{}, les opérateurs binaires \macro{divides}, \macro{ndivides} et \macro{modulo} ont été ajoutés comme alias respectifs de \macro{mid}, \macro{nmid} et \macro{bmod} qui sont proposés par le package \verb+amssymb+. Un opérateur \macro{nequiv} a été aussi ajouté.

\begin{latexex}
$10 \divides 150$ au lieu de
$10 | 150$

$10 \ndivides 154$ au lieu de
$10 \not| 154$

$a \nequiv b \modulo p
 \iff
 p \ndivides (a - b)$.
\end{latexex}


% ---------------------- %
\section{Fonctions nommées spéciales}

%\newparaexample*{}

Deux fonctions nommées \macro{pgcd} et \macro{ppcm} utiles au francophone ont été ajoutées ainsi que la fonction \macro{lcm} pour les anglophones car cette dernière n'est pas disponible par défaut.
La liste complète des fonctions nommées est donnée ci-dessous dans la section \ref{tnsarith-all-named-functions}.

\begin{latexex}
$\pgcd x = \gcd x$ et $\ppcm x = \lcm x$
\end{latexex}


% ---------------------- %
\section{Fractions continuées}

\subsection{Fractions continuées standard}

%\newparaexample*{}

Dans l'exemple suivant, la notation en ligne semble être due à Alfred Pringsheim. La notation à gauche utilise toujours le maximum d'espace pour améliorer la lisibilité.

\begin{latexex-flat}
 $\contfrac {u_0 | u_1 | u_2 | \dots | u_n}
= \contfrac*{u_0 | u_1 | u_2 | \dots | u_n}$
\end{latexex-flat}


% ---------------------- %


\subsection{Fractions continuées généralisées}

%\newparaexample*{}

Voici comment écrire une fraction continuée généralisée.

\begin{latexex-flat}
 $\displaystyle
  \contfracgene {a | b | c | d | e | f | \dots | y | z}
= \contfracgene*{a | b | c | d | e | f | \dots | y | z}$
\end{latexex-flat}


% ---------------------- %


\subsection{Comme une fraction continuée isolée}

%\newparaexample*{}

La raison d'être de la macro ci-dessous vient juste de son usage en interne.

\begin{latexex}
$\singlecontfrac{a}{b}$
pour les fous\dots :-)
\end{latexex}


% ---------------------- %


\subsection{\texorpdfstring{L'opérateur $\contfracope$}%                           {L'opérateur K}}
                           {L'opérateur K}}

\newparaexample{}

La notation suivante est proche de celle qu'utilisait Carl Friedrich Gauss.

\begin{latexex-flat}
 $\displaystyle
  \contfracope_{k=1}^{n} (b_k:c_k)
= \cfrac{b_1}%
        {\contfracgene{c_1 | b_2 | c_2 | b_3 | \dots | b_n | c_n}}$
\end{latexex-flat}


\begin{remark}
    La lettre $\contfracope $vient de "kettenbruch" qui signifie "fraction continuée" en allemand.
\end{remark}


% ---------------------- %


\newparaexample{}

\begin{latexex-flat}
 $\displaystyle
  u_0 + \contfracope_{k=1}^{n} (1:u_k)
= \contfrac{u_0 | u_1 | u_2 | \dots | u_n}$
\end{latexex-flat}


% ---------------------- %
\newpage

\section{Historique}

Nous ne donnons ici qu'un très bref historique récent
\footnote{
	On ne va pas au-delà de un an depuis la dernière version.
}
de \verb+tnsarith+ à destination de l'utilisateur principalement.
Tous les changements sont disponibles uniquement en anglais dans le dossier \verb+change-log+ : voir le code source de \verb+tnsarith+ sur \verb+github+.

\begin{description}
% Changes shown - START

    \medskip
    \item[2020-07-12] Nouvelle version mineure \verb+0.1.0-beta+. 
        
    \begin{itemize}[itemsep=.5em]
        \item \topic*{Fonctions nommées} ajout de \macro{pgcd}, \macro{ppcm} et \macro{lcm}.
    \end{itemize}
    
    \separation

% ------------------------ %

    \medskip
    \item[2020-07-10] Première version \verb+0.0.0-beta+.
% ------------------------ %

% Changes shown - END 
\end{description}


\newpage
\section{Toutes les fiches techniques} \label{techincal-ids}









\subsection{Opérateurs de base}



\IDope{divides}

\IDope{ndivides}

\extraspace

\IDope{nequiv}

\extraspace

\IDope{modulo}


\subsection{Fonctions nommées spéciales}



% List of functions without parameter - START

\foreach \k in {pgcd, ppcm}{

    \IDope{\k}
}
                
\separation

\foreach \k in {lcm}{

    \IDope{\k}
}

% List of functions without parameter - END


\subsection{Fractions continuées}

\subsubsection{Fractions continuées standard}



\IDmacro[a]{contfrac }{1}

\IDmacro[a]{contfrac*}{1}

\IDarg{} tous les éléments de la fraction continuée séparés par des \verb+|+.


% ---------------------- %


\subsubsection{Fractions continuées généralisées}



\IDmacro[a]{contfracgene }{1}

\IDmacro[a]{contfracgene*}{1}

\IDarg{} tous les éléments de la fraction continuée généralisée séparés par des \verb+|+.


% ---------------------- %


\subsubsection{Comme une fraction continuée isolée}



\IDmacro[a]{singlecontfrac}{2}

\IDarg{1} le pseudo numérateur.

\IDarg{2} le pseudo dénominateur.


% ---------------------- %


\subsubsection{\texorpdfstring{L'opérateur $\contfracope$}%
                           {L'opérateur K}}



La macro suivante sans argument a un comportement spécifique vis à vis des mises en index et en exposant. 


\separation


\IDope{contfracope}




\end{document}
